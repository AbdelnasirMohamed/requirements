\documentclass{template/openetcs_article}
\usepackage[utf8x]{inputenc}
\usepackage{color}
\usepackage{lipsum,url}
\graphicspath{{./template/}{.}{./images/}}
\begin{document}
\frontmatter
\project{openETCS}

\newcommand{\FIXME}[1]{\marginpar{FIXME}\textsf{FIXME: #1}}

%Please do not change anything above this line
%============================
% The document metadata is defined below

%assign a report number here
\reportnum{OETCS/WP2/D01}

%define your workpackage here
\wp{Work-Package 2: ``Requirements''}

%set a title here
\title{SRS subset for modelling tool benchmarking}

%set a subtitle here
%\subtitle{Revision}

%set the date of the report here
\date{November 2012}
%\date{\today}

%define a list of authors and their affiliation here

% alphabetical order
\author{David Mentré}

\affiliation{Mitsubishi Electric R\&D Centre Europe}

\author{Stanislas Pinte}

\affiliation{ERTMS Solution}

\author{Guillaume Pottier}

\affiliation{SNCF}

\author{WP2 participants}

\affiliation{OpenETCS}

% define the coverart
\coverart[width=350pt]{chart}

%define the type of report
\reporttype{Requirements}


\begin{abstract}
%define an abstract here

This document defines the subset of SRS SUBSET-026 that should be used
to evaluate modelling tools.

\end{abstract}

%=============================
%Do not change the next three lines
\maketitle
\tableofcontents
%\listoffiguresandtables
\newpage
%=============================

% The actual document starts below this line
%=============================


%Start here





% Makes Marginpars easier to read
\setlength{\marginparwidth}{1in}
\let\oldmarginpar\marginpar
\renewcommand\marginpar[1]{\-\oldmarginpar[\raggedleft\scriptsize #1]%
{\raggedright\scriptsize #1}}

\newcommand{\oldtext}[1]{{Old: \scriptsize #1}}

\newenvironment{inoutput}
{\vspace{2mm}
\noindent
\begin{tabular}{|r|p{.68\linewidth}|l|}
\hline}
{
\hline
\end{tabular}}

\section{Introduction}

One goal of openETCS is to make a model of the ERTMS/ETCS System
Requirement Specification (SRS). Several tools are possible to make
this model. In order to evaluate them, we need to define a subset of
the SRS that would be modelled by each tool, therefore allowing to
compare the tools on the same basis.

This document defines this subset of SRS.

\section{SRS Subset definition}

The following paragraphs of UNISIG SUBSET-026 v3.3.0 should be used
in the benchmarking model in order to evaluate a tool:

\begin{description}
\item [§3.5.3 Establishing a communication session]

  Rationale: Sample of the communication part.

\item [§3.6.3.2 Location, Continuous Profile Data and Non-Continuous
  Profile Data]

  Rationale: example of complex generic data structure.

\item [§3.8.3 Structure of Movement Authority and §3.8.5 Update of
  Movement Authority]

  Rationale: example of complex procedure, with complex data.

\item [§3.11.3 Static Speed Profile and §3.11.12 Gradients]

  Rationale: example of data structure, referring to §3.6.3.2 and used
  by §3.13.4.

\item [§3.13 ??] \FIXME{We should find a representative subset of
  §3.13. Guillaume proposes §3.13.4 (Acceleration / Deceleration due
  to gradients). Stanislas and David think this is not enough.}

\item [§4.6.2 (Transitions Table) and §4.6.3 (Transitions Condition
  Table)] Only transitions:
  \begin{enumerate}
  \item from SB to SH
  \item from SB to FS
  \item from SB to IS
  \end{enumerate}

  Rationale: Having transitions at different priority level is
  important to look at priority issues and exclusion issues at the
  same priority level.

\item [§4.8.3.2 From National System X (through STM interface)]

  Rationale: Model a small table. \FIXME{Isn't such a table redundant
    with §4.6.2?}

\item [§5.9 Procedure On-Sight] \FIXME{§5.6 also proposed by
  Guillaume}

 Rationale: Procedure sample, contains a timer. Procedure not too long
 compared to Start of Mission.

\item [§8.7.2 Movement Authority message] This includes reference to
  Packet 15 (§7.4.2.4). \FIXME{Maybe reference one optional packet}

  Rationale: That would be a perfect use case for tools able to model
  things down to bit level.
\end{description}

\section{Other open questions}

\FIXME{Should we model an API? E.g. Odometer? Which reference
  document?}

\end{document}
Local Variables:
ispell-local-dictionary: "english"
End:

% LocalWords:  SRS ERTMS ETCS
*)

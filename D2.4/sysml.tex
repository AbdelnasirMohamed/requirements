
\section{SysML approach with Papyrus}

\subsection{Introduction to SysML with Papyrus}

\begin{comment}
short introduction to sysML language with references to  OMg specification and major document which describe it + short description of main concepts of SysML +  short description of particularities of SysML on papyrus. (parts of 07.1.3\_O7.1.7 may be reused). 
\end{comment}

\subsection{SysML in the project}

\begin{comment}
for high level modelling to complete.

First ideas 

  \begin{itemize}
  	\item SSRS:  
  	\begin{itemize}
  		\item functional architecture (organic architecture)
  		\item logical interfaces 
  		\item requirements: Todo but may be completed with other tools as ProR)
  		\item data dictionary (how to access a common data repository)
  		\item sequence of functions ?
  		\item need to declare specific domain objects (switch, signals,...) ?
	\end{itemize} 
	\item API: 
  	\begin{itemize}
  		\item logical interface 
  		\item sequence of functions ?
  		\item implementation requirements
	\end{itemize}  	
	\item Sub-system semi-formal definition
	  	\begin{itemize}
  		\item sequence of functions 
  		\item activities
  		\item state machine
	\end{itemize}  	
    \item Safety: 
  	\begin{itemize}
  		\item Functional Breakdown Structure (common with SSRS)
  		\item profile for safety
	\end{itemize}  	
    \item VnV : test case

  \end{itemize}

\end{comment}

\subsection{Selection of used diagrams}

SysML diagrams that can be used for modeling are:
\begin{itemize}
\item \emph{Package Diagram}: Structure description
\item \emph{Block Definition Diagram} (BDD): Structure description
\item \emph{Internal Block Diagram} (IBD): Structure description
\item \emph{State Machine Diagram}: Behaviour description
\item \emph{Requirement Diagram}: Requirements description
\end{itemize}

The following other SysML diagrams \emph{cannot} be used:
\begin{itemize}
\item \emph{Parametric Diagram}
\item \emph{Activity Diagram}
\item \emph{Sequence Diagram}
\item \emph{Use Case Diagram}
\end{itemize}

\subsection{Restrictions on Package Diagram}

Note: The naming of the nodes and paths of this section and the
following one is the same as Annex A of book ``A Practical Guide to
SysML''.


The following Package Diagram nodes and paths can be used for
modelling:
\begin{itemize}
\item Comment Note
\item Package Node
\item Packageable Element Node
\end{itemize}

The following items are \emph{cannot} be used:
\begin{itemize}
\item Model Node
\item View Node
\item Viewpoint Node
\item Containenement Path
\item Import Path
\item Dependency Path
\item Conform Path
\item Metamodel Node
\item Metaclass Node
\item Model Library Node
\item Stereotype Node
\item Profile Node
\item Generalization Path
\item Extension Path
\item Associatin Path
\item Reference Path
\item Profile Application Path
\end{itemize}

\subsection{Restrictions on Block Definition Diagram}

The following Block Definition Diagram nodes and paths can be used for
modelling:
\begin{itemize}
\item Block Node
\item Enumeration Node
\item Composite Association Path
\item Flow Specification Node
\item Atomic Flow Port Node
\end{itemize}

The following items are \emph{cannot} be used:
\begin{itemize}
\item Quantity Kind and Unit Nodes
\item Value Type Node
\item Actor Node
\item Interface Block Node (SysML 1.3)
\item Interface Node
\item Signal Node
\item Interface Compartments for Block Node
\item Reference Association Path
\item Association Block Path and Node
\item Generalization Path
\item Full Port Node
\item Proxy Port Node
\item Proxy Port Node With Interfaces
\item Port Compartments for Block Node
\item Nonatomic Flow Port Node
\item Block Node with Constraint Compartment
\item Constraint Block Node
\item Activity Node
\item Activity Composition Path
\item Object Node Composition Path
\item Instance Specification Node
\item Association Instance Specification (Link) Path
\end{itemize}

\subsection{Restrictions on Internal Block Diagram}

The following Internal Block Diagram nodes and paths can be used for
modelling:
\begin{itemize}
\item Part Node
\item Connector Path
\end{itemize}

The following items are \emph{cannot} be used:
\begin{itemize}
\item Actor Part Node
\item Reference Node
\item Participant Property Node
\item Value Property Node
\item Connector Property Path and Node
\item Item Flow Node
\end{itemize}

\subsection{Restrictions on State Machine Diagram}

The following State Machine Diagram nodes and paths can be used for
modelling:
\begin{itemize}
\item State Machine with Entry- and Exit-Point Pseudostate Nodes
\item Atomic State Node
\item Composite State with Entry- and Exit-Point Pseudostate Nodes
\item Composite State Node with Multiple Region
\item Terminate Pseudostate Node
\item Initial Pseudostate Node
\item Final State Node
\item Action Node
\item Time Event Transition Path
\item Change Event Transition Path
\end{itemize}

The following items are \emph{cannot} be used:
\begin{itemize}
\item Sub-State Machine Node with Connection Points
\item Choice Pseudostate Node
\item Junction Pseudostate Node
\item Trigger Node
\item Send Signal Node
\item Join Pseudostate Node
\item Frok Pseudostate Node
\item History Pseudostate Node
\item Signal Event Transition Path
\item Call Event Transition Path
\end{itemize}

\subsection{Restrictions on Requirement Diagram}

The following Requirement Diagram nodes and paths can be used for
modelling:
\begin{itemize}
\item Requirement Node
\item Package Node
\item Containment Path
\item Derivation Path
\item Satisfaction Path
\item Verification Path
\item Rationale Callout
\item Problem Callout
\end{itemize}

The following items are \emph{cannot} be used:
\begin{itemize}
\item Requirement Related-Type Node
\item Trace Compartment
\item Test Case Node
\item Refinement Path
\item Trace Path
\item Copy Path
\item Trace Callout
\item Derivation Callout
\item Verification Callout
\item Statisfaction Callout
\item Refinement Callout
\item Master Requirement Callout
\end{itemize}



\subsection{Model patterns}



\begin{comment}
Example of patterns to use.

\end{comment}


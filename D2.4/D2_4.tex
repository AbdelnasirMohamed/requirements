\documentclass{template/openetcs_article}
% Use the option "nocc" if the document is not licensed under Creative Commons
%\documentclass[nocc]{template/openetcs_article} 
\usepackage{rotating,url,color}
\graphicspath{{./template/}{.}{./images/}}
\begin{document}
\frontmatter
\project{openETCS}

%Please do not change anything above this line
%============================
% The document metadata is defined below

%assign a report number here
\reportnum{OETCS/WP2/D2.4.0~--~00/00}

%define your workpackage here
\wp{Work-Package 2: ``Definition''}

%set a title here
\title{OpenETCS methods}

%set a subtitle here
\subtitle{ Definition of the methods used to perform the formal description}

%set the date of the report here
\date{February 2013}

%define a list of authors and their affiliation here

\author{Marielle Petit-Doche}
\affiliation{Systerel}
  
  
% define the coverart
\coverart[width=350pt]{chart}

%define the type of report
\reporttype{Definition}


\begin{abstract}
This document give a description of the process to be applied in the OpenETCS project. It gives a description of the activities to specify and design a critical system in a first part. The second part presents an abstract description of the case study issued from subset 26.


Detailed process in regards of state of the art results and Cenelec standard requirements  (detailed design plan)
- For each step  of the process defined in T.2.2.1 :
           - objectives of the step
           - Input/output
           - proposed techniques
- rules on languages and tools

\end{abstract}

%=============================
%Do not change the next three lines
\maketitle
\tableofcontents
\listoffiguresandtables
\newpage
%=============================

% The actual document starts below this line
%=============================


%%%%%%%%%%%%%%%%%%%%%%%%%%%%%%%%%%%%%%%%%%%%%%%%%%%%%%%%%%%%%%
%%%              My macros (=> Sylvain Baro)               %%%
%%%%%%%%%%%%%%%%%%%%%%%%%%%%%%%%%%%%%%%%%%%%%%%%%%%%%%%%%%%%%%
\newcommand{\tbd}{\colorbox{cyan}{\%\%To Be Defined\%\%}}
\newcommand{\tbc}{\colorbox{cyan}{\%\%To Be Confirmed\%\%}}
\newcommand{\todo}[1]{\colorbox{cyan}{\%\%{#1}\%\%}}
\newlength{\origindent}

\newenvironment{issue}{
	\begin{quote}
	\begin{itshape}Open Issue. 
}{
	\end{itshape}
	\end{quote}
}

\newenvironment{comment}{
	\begin{quote}
	\begin{itshape}Comment. 
}{
	\end{itshape}
	\end{quote}
}

\newenvironment{justif}{
	\begin{quote}
	\begin{itshape}Justification. 
}{
	\end{itshape}
	\end{quote}
}
%% Requirements.


\newcounter{reqnum}
\setcounter{reqnum}{0}
\newcounter{subreqnum}
\newcounter{subsubreqnum}
\newlength{\partopbuf}
\newlength{\topbuf}

% Automated numbering versions of the macros
\newcommand{\req}[1]{\addtocounter{reqnum}{1} \setcounter{subreqnum}{0}
	\begin{description}\item[{\small\reqt-X-\thereqnum}] #1\end{description}
}

\newcommand{\subreq}[1]{
	\addtocounter{subreqnum}{1} \setcounter{subsubreqnum}{0}
	\addtolength{\leftmargini}{1cm}
	\begin{description}
	\item[\hspace{0.5cm}{\small\reqt-X-\thereqnum.\thesubreqnum}] #1
	\end{description}
	\addtolength{\leftmargini}{-1cm}
}

\newcommand{\subsubreq}[1]{
	\addtocounter{subsubreqnum}{1}
	\addtolength{\leftmargini}{2cm}
	\begin{description}
	\item[\hspace{1cm}{\small\reqt-X-\thereqnum.\thesubreqnum.\thesubsubreqnum}] #1
	\end{description}
	\addtolength{\leftmargini}{-2cm}
}

% Fixed version of the commands
\newcommand{\reqfixed}[3]{\addtocounter{reqnum}{1} \setcounter{subreqnum}{0}
	\begin{description}\item[{\small\reqt-#1-#2}] #3\end{description}
}

\newcommand{\subreqfixed}[4]{
	\addtocounter{subreqnum}{1} \setcounter{subsubreqnum}{0}
	\addtolength{\leftmargini}{1cm}
	\begin{description}
	\item[\hspace{0.5cm}{\small\reqt-#1-#2.#3}] #4
	\end{description}
	\addtolength{\leftmargini}{-1cm}	
}

\newcommand{\subsubreqfixed}[5]{
	\addtocounter{subsubreqnum}{1}
	\addtolength{\leftmargini}{2cm}
	\begin{description}
	\item[\hspace{1cm}{\small\reqt-#1-#2.#3.#4}] #5
	\end{description}
	\addtolength{\leftmargini}{-2cm}	
}

% Citation of the requirement

% Citation of the reference (for markup purpose)
%\newcommand{\refreq}[1]{\textbf{#1}}

% Citation of the reference and text (for markup purpose)
% The purpose of this is to automatically replace the placeholder by the 
% full text. \fullrefreq{R-xxx}{} or \fullrefreq{R-xxx}{blabla} 
% will be replaced by \fullrefreq{R-xxx}{text of the R-xxx requirement} 
%\newcommand{\fullrefreq}[2]{\textbf{#1}: \textrm{#2}}


\def\reqt{R-WP2/D2.3.0}
% Start here
\section{Introduction}

The purpose of this document is to describe, for the OpenETCS project, the activties of specification and design. However the activities of verification and validation are not in the scope of this document and will be described in WP4/D4.x.x.

These activities shall follow the requirements of EN 50126 and EN 50128 and reflect usual activities for the development of railway critical systems (see D2.1.0  and D2.2.0).

This document  contents two parts :
\begin{itemize}
\item the description of the process to applied in the OpenETCS project
\item the abstract description and the limit of the case study 
\end{itemize}

\section{Reference documents}
\begin{itemize}
\item CENELEC EN 50126-1 --- 01/2000 --- \emph{Railways applications –- The specification and 
demonstration of Reliability, Availability, Maintenability and Safety (RAMS) –- Part 1: 
Basic requirements and generic process}
\item CENELEC EN 50128 --- 10/2011 --- \emph{Railway applications -- Communication, signalling and 
processing systems -- Software for railway control and protection systems}
\item CENELEC EN 50129 --- 05/2003 --- \emph{Railway applications –- Communication, signalling and 
processing systems –- Safety related electronic systems for signalling}
\item FPP --- \emph{Project Outline Full Project Proposal Annex OpenETCS} -- v2.2
\item SUBSET-026 3.3.0 --- \emph{System Requirement Specification}
\item SUBSET-076-x 2.3.y --- Test related ERTMS documentation
\item SUBSET-088 2.3.0 --- \emph{ETCS Application Levels 1 \& 2 - Safety Analysis}
\item SUBSET-091 2.5.0 --- \emph{Safety Requirements for the Technical Interoperability
of ETCS in Levels 1 \& 2}
\item CCS TSI --- \emph{ CCS TSI for HS and CR transeuropean rail has been adopted by a Commission Decision 2012/88/EU on the 25th January 2012}
\end{itemize}

\section{Conventions}
The requirements are prefixed by “R-zz-x-y”, and are written in a roman typeface, where ``R'' 
stands for ``Requirement'', ``zz'' identifies the source document,``x'' 
is the version number and``y'' is the identifier of the requirement. All the text 
written in italics is not a requirement: it may be a note, an open issue, an 
explanation of the requirements, or an example.

The placeholder “\todo{xxx}” is used to indicates that a paragraph or section is not finished, 
to be defined or to be confirmed.

\section{Glossary}
\begin{description}
\item[API] Application Programming Interface
\item[FME(C)A] Failure Mode Effect (and Criticity) Analysis
\item[I/O] Input/Output
\item[OBU] OnBoard Unit
\item[QA] Quality Analysis
\item[RBC] Radio Block Center
\item[RTM] RunTime Model
\item[SIL] Safety Integrity Level
\item[THR] Tolerable Hazard Rate
\item[V\&V] Verification \& Validation
\end{description}



\section{Short introduction on formal approaches to  design and validate critical systems}

\subsection{What is a formal approach?}

A \emph{formal} approach is a way to describe system or software
that builds upon (i) rigorous syntax and (ii) rigorous semantics.

The \emph{syntax} defines how the system or software description is
built and valid. It is usually made through a grammar and a set of
additional constraints. It can be textual or graphical.

The \emph{semantics} gives a meaning to each object found in the
system or software description. This meaning is given using a
mathematical model, i.e., use of mathematical objects attached to each
element of the syntax and mathematical rules that define how those
objects interacts with other objects. The mathematical models used can
be very different from one formal approach to another one. For example
the B~Method uses the Generalized Substitutions, SCADE relies on the
Synchronous language Lustre, etc. One should notice that being able to
compile or run a language is not enough to give it some semantics, as
this semantics is hidden within the execution/compilation steps. An
explicit document should be provided. This document can be informal
(e.g. the B-Book) or formal (BiCoq formalization of B~Method in Coq
formal language).

A \emph{semi-formal} approach is one where the syntax is precisely
defined but the semantics is not precisely defined, usually through some
English text. Typical semi-formal approaches are the Matlab language or
the SysML/UML formalisms.

A semi-formal approach can become formal if its semantics is
rigorously defined through a mathematical model.

\subsection{When are formal approaches recommended according to CENELEC standard?}

The use of formal approaches is \emph{Highly Recommended} for SIL3 and
SIL4 software according to CENELEC EN 50128:2011.

\subsection{Which constraints are required on the use of formal approaches?}

Each formal approach has some restriction on the kind of software or
system it can be applied to. Moreover, each formal approach is
specialized in the verification of some kind of property. Therefore a
formal approach should be chosen in accordance to the verification
objectives.

Moreover, using a formal approach can impact the overall system building
process. For example software developed using the B~Method follows a
specific process and imposes a very specific architecture, very different
from designing C software. In the same way, the usage of a formal
approach can impose specific resource needs at different phases of the
project lifetime. For example, more work on the requirement analysis and
formalization phase.

Last but not least, as a formal approach brings its benefits only inside
a given boundary, the development process should be designed to transfer
these benefits beyond those boundaries. For example, code compilation of
a verified source code should be done in such a way as to ensure that
the verified properties are kept in the compiled code.

\subsection{Which are the benefits to use formal approaches?}

Several benefits are expected from the use of formal approaches.

The first benefit is to enhance the understanding of the formalized
system or software. By using a non ambiguous notation, the designer is
forced to clarify his mind. Very often, several design issues or defects
are found at this step, and in general, fixing errors at this step is
much less costly than in later development phases.

The second benefit is to enable the verification of some properties in
an exhaustive way. Therefore avoidance of certain kinds of bugs can be
guaranteed. Of course, such guarantee can only be obtained if the
formal method is used along some specific way and on a well delimited
part of the software and system (for example one cannot guarantee
properties on variables outside program boundary).

The third benefit is to allow Correct by Construction software or
system building. By verifying properties along the construction
cycle of a system or software, one can ensure that some formalized
requirements are fulfilled in the final software. For example, one can
ensure that some variables stay in well defined boundaries.

The fourth benefit is the ability to easily extend the formalized
system or software, by updating the formal description. After such an
update, applying the formal verification allows to know precisely
which parts are no longer valid and focus development effort on them,
without the need to re-verify parts not impacted by the change.

\subsection{How to use formal approaches?}

In the design and development of a system using an approach based on
formal methods, there are two orthogonal aspects to consider: at which
stage (or stages) in the development cycle the formal approach will be
used and how it will be used, i.e., choice of approach, technical
realization.

In the development cycle, there are three main stages where a formal
approach can be applied:

\begin{itemize}
\item Formalization of Requirements
\item Design Support
\item Implementation Verification
\end{itemize}

\subsubsection{Formalization of Requirements}
\label{sec:formalization-of-req}

In the System Development Phase and Software Requirements Phase, a
formal approach applied to initial requirements can bring
clarifications, by enforcing a non-ambiguous meaning for all
parties. In case making such a formalization of requirements is
difficult, it usually triggers further clarification efforts between
involved parties.

\subsubsection{Design Support}
\label{sec:design-support}

In the Design and Architecture Phase, a formal approach can support the
system design and architecture design. In this phase, systematic errors
can be detected which can be very difficult and costly or even impossible
to fix later.

In combination with a refinement based correct by construction approach,
it is possible to have high level properties on the whole system which
are refined to sub-properties on the different parts of the system
architecture while designing the system. An example of such an approach
is the Event-B method.

\subsubsection{Implementation Verification}
\label{sec:impl-verif}

In the later phases of the development process, formal approaches can
deal with formal reasoning over the actual functional system source
code. Depending on the method, this code can be generated from a formal
model, derived via a refinement based approach or written manually,
annotated with formal properties.

Code generation from a higher level model is in particular interesting,
if the generator is qualified and code generation can reduce the required
testing of code. A refinement based approach will iteratively add detail
to a high level description until a detail level is reached which can be
implemented in programming languages, here often translation, i.e.,
side-by-side creation of refined model and source code is used. And
finally it is possible to manually write code which is annotated with
properties that can be verified formally (see also
Section~\ref{sec:contr-based-appr}. An example for a code generation
based approach is SCADE, the B~method is based on refinement and formal
proof and Frama-C, GNATprove / SPARK are based on source code annotation.






\section{Formal approaches for the design and development of a system}


\begin{comment}
This section will describe how formal methods are involved in the design of critical system.
\end{comment}



\subsection{System level model}

\subsection{Software level model}

\subsection{Functional architecture}

\subsection{Safety properties expression}

\req{The model-level safety properties shall be written in a formal language.}


\section{Formal approaches for V\&V of a critical system}

The various approaches previously presented can be used to check
various kind of properties:
\begin{itemize}
\item safety properties: ensure the system is safe;
\item functional properties: ensure the system works as expected
  regarding its functional behavior;
\item non-functional properties: ensure the system works has expected
  regarding its speed, capacity, ...
\end{itemize}

Due to the cost and complexity of formal analysis, use of formal
methods in the railway domain is usually focused on ensuring only
safety properties. We only consider them is the remaining of this
section.


\subsection{Formal approaches for verification}

Ensuring safety properties using formal approches starts in a similar
way to classical approaches. A safety analysis will produce the
properties that should be ensured to guarantee safe operation of the
system.

Usually such safety properties are high level properties (e.g., ``two
trains do not collide''). In order to be amenable to formal verification,
they should be partitioned into properties related to the system state
(e.g., ``there exists two free blocks between any occupied blocks'', ...)
and properties for specific system parts. Several system-related safety
properties can be associated to a single high-level safety property. In
general it should be verified that as a whole the partial properties
imply the high level properties.

Then those properties are checked to be valid, i.e., in any system
state a safety property is always true. This can be done with the
various approaches presented previously.

If a Correct by Construction or refinement based approach is used, some
traditional verification activities like unit or integration tests can be
avoided because they are ensured by the formal approaches. In this case
the high level property is refined side by side with the system model,
iteratively proving the correctness of the refinement steps.

\subsection{Validation of formal approaches}

Proving that a safety property is always valid on a system model does
not ensure the property is valid in the real life. Discrepancies can
occur between the system model and the real system. Moreover, errors
can occur during the formalization of the high-level safety property
into a set of system-related properties.

Therefore a validation activity is needed. Validation checks that
formalization of properties is correct, as well as all related
assumptions.

This is done using non formal techniques:
\begin{itemize}
\item Review;
\item Simulation and animation;
\item Test.
\end{itemize}

\subsection{Formal approaches for safety}

\begin{comment}
proof of safety requirements, static analysis, safety analysis, traceability,...
\end{comment}







\section{Guidelines on the approaches used for OpenETCS}


\begin{comment}
This section will be written for the final version of the document, after the approach and tools tio  use during the project will be selected.
\end{comment}

According to  the WP7 decision meeting, the 4th of July, in Paris, SysML, supported by the Papyrus tool,  has been chosen to  cover the highest level of modelling.

The choice of the approaches for the lower levels of modelling is not yet fixed.

This section gives a proposal on how to use the selected approaches to produce from the input documents (ERA documentation and complements)  to a SIL4 code.

\subsection{Sum up of chosen approaches}
\begin{comment}
list of chosen approaches, and for which activities they are used
\end{comment}


\subsection{Artifacts and common items}

\begin{comment}
list of artifacts used and provided by each approach; common items (data model); types
\end{comment}

\subsection{Name convention}


\begin{comment}
How to name object ? base : subset 26 §7.3.2:
" 7.3.2.11 All Variables have one of the following prefixes:
\begin{itemize}
\item A\_ Acceleration
\item D\_ distance
\item G\_ Gradient
\item L\_ length
\item M\_ Miscellaneous
\item N\_ Number
\item NC\_ class number
\item NID\_ identity number
\item Q\_ Qualifier
\item T\_ time/date
\item V\_ Speed
\item X\_ Text
\end{itemize}

Case sensitive language, keywords of target language (SysML, B, C, Scade 5?),...)
\end{comment}



%%%%%%%%%%%%%%%%%%%%%%%%%%%

%% Bibliography
\nocite{*}
\bibliographystyle{unsrt}
\bibliography{erdc}



% \begin{thebibliography}{9}

% \bibitem{lamport94}
  % Leslie Lamport,
  % \emph{\LaTeX: A Document Preparation System}.
  % Addison Wesley, Massachusetts,
  % 2nd Edition,
  % 1994.

% \end{thebibliography}

%===================================================
%Do NOT change anything below this line

\end{document}



\section{Short introduction on formal approaches to  design and validate critical systems}

\subsection{What is a formal approach ?}
semi-formal, formal, syntax, semantics,...

\subsection{When formal approaches are recommended according to CENELEC standard ?}

\subsection{Which constraints are required on the use of formal  approaches ?}


\subreq{The techniques applied to the software will be compliant regarding the SIL.}


\subreq{The tools used shall be developed in order to be certifiable according to EN 50128.}

\begin{comment}
No requirement on the way of doing this. \emph{E.g.} to have a certified 
(certifiable?) code generator, two generators and comparison of the result, one
 generation and one verification chain\dots
 \end{comment}
 
\subsection{Which are the benefits to use formal approaches ? }


\begin{comment}
From D.2.5 :

The purpose of the formalization is:
\begin{itemize}
\item to enhance the understanding of modelled subset;
\item to allow formal analysis of the modelled subset;
\item to be able to animate the model for testing an analyzing purpose;
\item to provide information on the completeness and soundness of the SUBSET-26;
\item to be used as a reference formal specification for the implementation of an OBU 
(by the OpenETCS project team and by industrial actors);
\item \dots
\end{itemize}
\end{comment}

\subsection{What means are involved behind a formal approach ?}

\subsubsection{Model edition}

\subsubsection{Mathematical analyses}
static checking, proof, model-checking, ...

\subsubsection{Simulation and code generation}


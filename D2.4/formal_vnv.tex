
\section{Formal approaches for V\&V of a critical system}

The various approaches previously presented can be used to check
various kind of properties:
\begin{itemize}
\item safety properties: ensure the system is safe;
\item functional properties: ensure the system works as expected
  regarding its functional behavior;
\item non-functional properties: ensure the system works has expected
  regarding its speed, capacity, ...
\end{itemize}

Due to the cost and complexity of formal analysis, use of formal
methods in the railway domain is usually focused on ensuring only
safety properties. We only consider them is the remaining of this
section.


\subsection{Formal approaches for verification}

Ensuring safety properties using formal methods starts in a similar
way to classical approaches. A safety analysis will produce the
properties that should be ensured to guarantee safe operation of the
system.

Usually such safety properties are high level properties (e.g., ``two
trains do not collide''). In order to be amenable to formal verification,
they should be partitioned into properties related to the system state
(e.g., ``there exists two free blocks between any occupied blocks'', ...)
and properties for specific system parts. Several system-related safety
properties can be associated to a single high-level safety property. In
general it should be verified that as a whole the partial properties
imply the high level properties.

Then those properties are checked to be valid, i.e., in any system
state a safety property is always true. This can be done with the
various approaches presented previously.

If a Correct by Construction or refinement based approach is used, some
traditional verification activities like unit or integration tests can be
avoided because they are ensured by the formal approaches. In this case
the high level property is refined side by side with the system model,
iteratively proving the correctness of the refinement steps.

\subsection{Validation of formal approaches}

Proving that a safety property is always valid on a system model does
not ensure the property is valid in the real life. Discrepancies can
occur between the system model and the real system. Moreover, errors
can occur during the formalization of the high-level safety property
into a set of system-related properties.

Therefore a validation activity is needed. Validation checks that
formalization of properties is correct, as well as all related
assumptions.

This is done using non formal techniques:
\begin{itemize}
\item Review;
\item Simulation and animation;
\item Test.
\end{itemize}

\subsection{Formal methods for safety}

\begin{comment}
proof of safety requirements, static analysis, safety analysis, traceability,...
\end{comment}




